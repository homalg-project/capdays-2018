\begin{frame}
 \frametitle{}
 \tableofcontents
\end{frame}

\section{Classical diagram chases}


% What are diagram chases?
\begin{frame}[fragile]
 \frametitle{What are diagram chases?}
 \pause
 \begin{block}{}
  Diagram chases are a tool in homological algebra
  used for proving 
  \pause
  \begin{enumerate}
   \item properties
   \pause
   \item {\only<7->{\color{red}}the existence}
   \pause
  \end{enumerate}
  of morphisms 
  \pause
  situated in (commutative) diagrams of prescribed shape.
 \end{block}
\end{frame}

% Connecting homomorphism in the snake lemma
\begin{frame}[fragile]
 \frametitle{Connecting homomorphism in the snake lemma}
 \begin{center}
 \begin{tikzpicture}[mystyle/.style={scale=.7}]
  \matrix[matrix of math nodes,column sep={70pt,between origins},row sep={40pt,between origins}] (s)
  { & & & |[name=Kernel]| \text{\only<1>{$\kernel(\gamma)$}\only<2>{${\color{red}c} \in \kernel(\gamma)$}\only<3->{$c \in \kernel(\gamma)$ }} & \\
    &|[name=A]| A &|[name=B]| \text{\only<1-3>{$B$}\only<4>{${\color{red}b} \in B$}\only<5->{$b \in B$}} &|[name=C]| \text{\only<1-2>{$C$}\only<3>{${\color{red}c} \in C$}\only<4->{$c \in C$}} &|[name=01]| 0 \phantom{C}  \\
    |[name=02]| \phantom{a'} 0 &|[name=A']| \text{ \only<1-5>{$A'$}\only<6>{${\color{red}a'} \in A'$}\only<7->{$a' \in A'$} } &|[name=B']| \text{ \only<1-4>{$B'$}\only<5>{${\color{red}b'} \in B'$}\only<6->{$b' \in B'$} } &|[name=C']| C' \\
    & |[name=Coker]| \text{ \only<1-6>{$\coker(\alpha)$}\only<7>{${\color{red}a' + \im( \alpha )}\in \coker( \alpha )$}\only<8->{$a' + \im( \alpha )\in \coker( \alpha )$} } & & & \\
  };
              \path[->,thick] (Kernel) edge (C);
              \path[->,thick] (C) edge (01);
              \path[->,thick] (A) edge (B);
              \path[->,thick] (B) edge node[mystyle,anchor=south] {$\epsilon$} (C);
              \path[->,thick] (A) edge node[mystyle,anchor=east] {$\alpha$} (A');
              \path[->,thick] (B) edge node[mystyle, anchor=east] {$\beta$} (B');
              \path[->,thick] (C) edge node[mystyle,anchor=east] {$\gamma$} (C');
              \path[->,thick] (02) edge (A');
              \path[->,thick] (A') edge node[mystyle,anchor=north] {$\mu$} (B');
              \path[->,thick] (B') edge (C');
              \path[->,thick] (A') edge (Coker);
              \visible<8->{
              \draw[->,rounded corners,thick,blue] (Kernel) -| ($(01.east)+(.5,0)$) |- ($(B)!.35!(B')$) -|
    ($(02.west)+(-.5,0)$) |- (Coker);}
  \end{tikzpicture}
 \end{center}
 
 \begin{block}{}
 \begin{center}
  \only<1>{Wanted: $\kernel(\gamma) \stackrel{\partial}{\longrightarrow} \coker( \alpha )$. \phantom{$\stackrel{:}{T_g}$} }
  \only<2>{Start: $c \in \kernel( \gamma )$. \phantom{$\stackrel{:}{T_g}$}}
  \only<3>{This lies in $C$. \phantom{$\stackrel{:}{T_g}$}}
  \only<4>{Choose: $b \in \epsilon^{-1}(\{c\})$. \phantom{$\stackrel{:}{T_g}$}}
  \only<5>{Map: $b \stackrel{\beta}{\mapsto} b'$. \phantom{$\stackrel{:}{T_g}$} }
  \only<6>{Compute: $a' \in \mu^{-1}(b')$. \phantom{$\stackrel{:}{T_g}$} }
  \only<7>{Map: $a' \mapsto a' + \im( \alpha )$.\phantom{$\stackrel{:}{T_g}$}}
  \only<8-9>{ Result: $c \stackrel{{\color{blue}\partial}}{\mapsto} a' + \im( \alpha )$.
  \visible<9>{\textbf{Context}: modules} \phantom{$\stackrel{:}{T_g}$}}
  \end{center}
 \end{block}
\end{frame}


% Classical solutions: embedding theorems
\begin{frame}[fragile]
\frametitle{Classical solutions: embedding theorems}
 \pause
 \begin{block}{Freyd-Mitchell embedding theorem}
  \pause
  Any small abelian category $\mathbf{A}$ admits an exact fully faithful covariant embedding 
  \[
   F: \mathbf{A} \hookrightarrow R-\mathbf{mod}
  \]
  into the category of $R$-modules for some ring $R$.
 \end{block}
 \pause
 \begin{block}{Application: existence of morphisms}
  \begin{center}
   \begin{tikzpicture}
   \coordinate (r) at (4,0);
   \coordinate (d) at (0,-1);
   \node (A) {$\Hom_{\mathbf{A}}(A,B)$};
   \node (B) at ($(A) + (r)$) {$\Hom_{R-\mathbf{mod}}(FA,FB)$};
   \visible<6->{
   \node (a) at ($(A) + (d)$) {$F^{-1}\phi$};
   }
   \visible<5->{
   \node (b) at ($(B) + (d)$) {$\phi$};
   }
   
   \draw[draw=none] (A) -- node[]{$\cong$} (B);
   \visible<6->{
   \draw[draw=none] (a) -- node[rotate=90]{$\in$} (A);
   \draw[draw=none] (a) -- node[]{$\leftrightarrow$} (b);
   }
   \visible<5->{
   \draw[draw=none] (b) -- node[rotate=90]{$\in$} (B);
   }
  \end{tikzpicture}
  \end{center}
 \end{block}
 \visible<7->{
  Problem: this isomorphism between $\Hom$-sets is \textbf{not constructive}.
  }
\end{frame}


\section{Additive relations}

% Back to the snake lemma
\begin{frame}[fragile]
 \frametitle{Back to the snake lemma}
 \begin{center}
 \begin{tikzpicture}[mystyle/.style={scale=.7}]
  \matrix[matrix of math nodes,column sep={70pt,between origins},row sep={40pt,between origins}] (s)
  { & & & |[name=Kernel]| \text{{$c \in \kernel(\gamma)$ }} & \\
    &|[name=A]| A &|[name=B]| \text{{\only<2->{\color{red}}$b \in B$}} &|[name=C]| \text{{$c \in C$}} &|[name=01]| 0 \phantom{C}  \\
    |[name=02]| \phantom{a'} 0 &|[name=A']| \text{ {$a' \in A'$} } &|[name=B']| \text{ {$b' \in B'$} } &|[name=C']| C' \\
    & |[name=Coker]| \text{ {$a' + \im( \alpha )\in \coker( \alpha )$} } & & & \\
  };
              \path[->,thick] (Kernel) edge (C);
              \path[->,thick] (C) edge (01);
              \path[->,thick] (A) edge (B);
              \path[->,thick] (B) edge node[mystyle,anchor=south] {$\epsilon$} (C);
              \path[->,thick] (A) edge node[mystyle,anchor=east] {$\alpha$} (A');
              \path[->,thick] (B) edge node[mystyle, anchor=east] {$\beta$} (B');
              \path[->,thick] (C) edge node[mystyle,anchor=east] {$\gamma$} (C');
              \path[->,thick] (02) edge (A');
              \path[->,thick] (A') edge node[mystyle,anchor=north] {$\mu$} (B');
              \path[->,thick] (B') edge (C');
              \path[->,thick] (A') edge (Coker);
              \draw[->,rounded corners,thick,blue] (Kernel) -| ($(01.east)+(.5,0)$) |- ($(B)!.35!(B')$) -|
    ($(02.west)+(-.5,0)$) |- (Coker);
  \end{tikzpicture}
 \end{center}
 
 \begin{block}{}
 \begin{center}
  \only<1>{ Result: $c \stackrel{{\color{blue}\partial}}{\mapsto} a' + \im( \alpha )$. \phantom{$\stackrel{:}{T_g}$} }
  \only<2>{ Crucial step: the \textbf{uncanonical} choice $b \in \epsilon^{-1}(\{c\})$. \phantom{$\stackrel{:}{T_g}$} }
  \only<3>{ Make this step canonical: \textbf{relations} instead of maps: $c \mapsto \epsilon^{-1}(\{c\})$ \phantom{$\stackrel{:}{T_g}$}}
  \end{center}
 \end{block}
\end{frame}

% Relations
\begin{frame}[fragile]
 \frametitle{Relations}
 Let $A,B$ be abelian groups.
 \pause
 \begin{block}{Definition}
  A subgroup $f \subseteq A \oplus B$ is called a \textbf{relation from $A$ to $B$}.
 \end{block}
 \pause
 \begin{block}{Example}
  Let $\epsilon: A \rightarrow B$ be a homomorphism of abelian groups. 
  \pause
  \[
   \Gamma( \epsilon ) := \{ (a,b) \in A \oplus B ~|~ \epsilon(a) = b \}
  \]
  is a relation from $A$ to $B$\pause, called \textbf{graph of $\epsilon$}\pause, and
  \[
   \epsilon^{-1} := \{ (b,a) \in B \oplus A ~|~ \epsilon(a) = b \}
  \]
  is a relation from $B$ to $A$\pause, called \textbf{pseudo-inverse of $\epsilon$}.
 \end{block}
\end{frame}

% Relations
\begin{frame}[fragile]
 \frametitle{Relations}
 \begin{block}{Composition of relations}
  \pause Given $f \subseteq A \oplus B$ and $g \subseteq B \oplus C$, define \pause
  \[
   g \circ f := \{ (a,c) \in A \oplus C ~|~ \exists b \in B: (a,b) \in f, (b,c) \in g \}
  \]
 \end{block}
 \pause
\begin{center}
 If $f$ and $g$ correspond to maps, this describes their usual composition.
\end{center}
\end{frame}

% Relations
\begin{frame}[fragile]
 \frametitle{Relations}
 \begin{itemize}
  \item[Q:] When does an additive relation $f \subseteq A \oplus B$ defines an honest map (a group homomorphism)?
 \end{itemize}
 \pause
 \begin{block}{Domain}
  $\domain(f) := \big\{ a \in A \mid \exists b \in B: (a,b) \in f \big\} \visible<5->{{\color{red}\ = A}}$
 \end{block}
 \pause
 \begin{block}{Defect}
  $\defect(f) := \big\{ b \in B \mid (0,b) \in f \big\} \visible<7->{{\color{red}\ = 0 }}$
 \end{block}
 \pause
 \begin{itemize}
  \item[A:] When it has a full domain \pause\pause and $0$ defect.
 \end{itemize}

\end{frame}

% The snake lemma for a last time
\begin{frame}[fragile]
 \frametitle{The snake lemma for a last time}
 
 \begin{center}
 \begin{tikzpicture}[mystyle/.style={scale=.7}]
  \matrix[matrix of math nodes,column sep={70pt,between origins},row sep={40pt,between origins}] (s)
  { & & & |[name=Kernel]| \kernel(\gamma) & \\
    &|[name=A]| A &|[name=B]| B &|[name=C]| C &|[name=01]| 0 \phantom{C}  \\
    |[name=02]| \phantom{a'} 0 &|[name=A']| A' &|[name=B']| B' &|[name=C']| C' \\
    & |[name=Coker]| \coker(\alpha) & & & \\
  };
              \only<1>{\path[->,thick] (Kernel) edge node[mystyle,anchor=west] {$\iota$} (C); }
              \only<2->{\path[->,blue,thick] (Kernel) edge node[mystyle,anchor=west] {$\iota$} (C); }
              
              \path[->,thick] (C) edge (01);
              \path[->,thick] (A) edge (B);
              
              \only<1->{\path[->,thick] (B) edge node[mystyle,anchor=south] {$\epsilon$} (C);}
              \only<3->{\path[->,red,thick, out=north west, in=north east] (C) edge node[mystyle,anchor=south] {$\epsilon^{-1}$} (B);}
              
              \path[->,thick] (A) edge node[mystyle,anchor=east] {$\alpha$} (A');
              
              \only<1-3>{\path[->,thick] (B) edge node[mystyle, anchor=east] {$\beta$} (B');}
              \only<4->{\path[->,blue,thick] (B) edge node[mystyle, anchor=east] {$\beta$} (B');}
              
              \path[->,thick] (C) edge node[mystyle,anchor=east] {$\gamma$} (C');
              \path[->,thick] (02) edge (A');
              
              \only<1->{\path[->,thick] (A') edge node[mystyle,anchor=north] {$\mu$} (B');}
              \only<5->{\path[->,red,thick, out=south west, in=south east] (B') edge node[mystyle,anchor=north] {$\mu^{-1}$} (A');}
              
              \path[->,thick] (B') edge (C');
              
              \only<1-5>{\path[->,thick] (A') edge node[mystyle,anchor=east] {$\pi$}(Coker);}
              \only<6->{\path[->,blue,thick] (A') edge node[mystyle,anchor=east] {$\pi$}(Coker);}
              \visible<7->{
              \draw[->,blue,rounded corners,thick] (Kernel) -| ($(01.east)+(.5,0)$) |- ($(B)!.35!(B')$) -|
    ($(02.west)+(-.5,0)$) |- (Coker);}
  \end{tikzpicture}
 \end{center}

 \begin{block}{}
 \begin{center}
  \only<1>{Wanted: $\kernel(\gamma) \stackrel{\partial}{\longrightarrow} \coker( \alpha )$. \phantom{$\stackrel{:}{T_g}$} }
  \only<2-6>{
  \visible<6->{${\color{blue}\pi} \circ$ }
  \visible<5->{${\color{red}\mu^{-1}} \circ$ }
  \visible<4->{${\color{blue}\beta} \circ$ }
  \visible<3->{${\color{red}\epsilon^{-1}} \circ$} 
  \visible<2->{${\color{blue}\iota}$ \phantom{$\stackrel{:}{T_g}$}}
  }
  \only<7->{\textbf{$\partial$ is an honest map given by a composition of relations!} \phantom{$\stackrel{:}{T_g}$}}
  \end{center}
 \end{block}
 
\end{frame}

\section{Generalized morphisms}

% From relations to generalized morphisms
\begin{frame}[fragile]
 \frametitle{From relations to generalized morphisms}
 \begin{block}{}
 \begin{center}
  \begin{itemize}
   \item \textbf{\color{blue}Wanted}: a categorical framework for relations.
   \pause
   \item \textbf{\color{blue}Solution}: generalized morphisms.
  \end{itemize}
 \end{center}
 \end{block} 
\end{frame}

% From relations to generalized morphisms
\begin{frame}[fragile]
 \frametitle{From relations to generalized morphisms}
 Let $A,B$ be objects in an abelian category $\mathbf{A}$.
 \pause
 \begin{block}{\visible<2->{Relation} \visible<5->{$\rightsquigarrow$ generalized morphism} \visible<6->{(data structure: span)}}
  \begin{center}
   \begin{tikzpicture}[mystyle/.style={scale=.7}]
  \matrix[matrix of math nodes,column sep={40pt,between origins},row sep={40pt,between origins}] (s)
  {
    |[name=A]| \text{ \visible<4->{$A$} } & |[name=AplusB]| \text{ \visible<2-3>{$A \oplus B$} } & |[name=B]| \text{ \visible<4->{$B$} } \\
    & |[name=D]| D & \\
  }; 
  \visible<2>{
  \path[right hook->, thick] (D) edge (AplusB);
  }
  \visible<3>{
  \path[right hook->, thick] (D) edge node[mystyle,anchor=east] {$\left( \begin{array}{cc} {\color{red}\alpha} & {\color{blue}\beta} \end{array} \right)$} (AplusB);
  }
  \visible<4->{
  \path[->, thick, red] (D) edge node[mystyle, anchor= east]{ ${\color{red}\alpha}$ }(A);
  \path[->, thick, blue] (D) edge node[mystyle, anchor= west]{ ${\color{blue}\beta}$ }(B);
  }
  \visible<5->{
  \path[->, dashed, thick] (A) edge (B);
  }
 \end{tikzpicture}
  \end{center}
 \end{block}
 \pause
 \pause
 \pause
 \pause
 \pause
 \begin{block}{Equality}
  \pause
  Two spans $( \alpha, \beta )$ and $( \alpha', \beta' )$ are \textbf{equal as generalized morphisms}
  if \pause
  \[
   \im\left( ( \alpha, \beta ): D \rightarrow A \oplus B \right) = \im\left( ( \alpha', \beta' ): D' \rightarrow A \oplus B \right).
  \]
 \end{block}
\end{frame}

% Composition of generalized morphisms
\begin{frame}[fragile]
 \frametitle{Composition of generalized morphisms}
 
 \begin{block}{Composition} \pause
 \begin{overlayarea}{\linewidth}{7.3\baselineskip}
 \begin{center}
   \begin{tikzpicture}
      \matrix (s) [matrix of math nodes,column sep={40pt,between origins},row sep={40pt,between origins}]
      {  A &  & \phantom{B}  &  & C \\
         & \phantom{X}  &  & \phantom{Y}  &  & \\ 
         & & \phantom{X \times_B Y} & & \\ };
      
      \visible<2-6>{
	\path[->,dashed,thick] (s-1-1) edge (s-1-3);
      }
      \visible<7->{
	\path[->,color=gray,dashed,thick] (s-1-1) edge (s-1-3);
	\path[-,dashed,thick] (s-1-1) edge (s-1-3);
      }
      \path[->,dashed,thick] (s-1-3) edge (s-1-5);
	
      \visible<2-6>{
	\path[->,thick] (s-2-4) edge (s-1-5);  
	\path[->,thick] (s-2-2) edge (s-1-1);
      }
      
      \visible<2-3>{
	\path[->,thick] (s-2-2) edge (s-1-3);
	\path[->,thick] (s-2-4) edge (s-1-3);
	\node at (s-2-2) {$X$};
	\node at (s-2-4) {$Y$};
	\node at (s-1-3) {$B$};
      }
	
      
      \visible<4-6>{
	  \path[->,color=red,thick] (s-2-2) edge (s-1-3);
	  \path[->,color=red,thick] (s-2-4) edge (s-1-3);
	  \node[color=red] at (s-1-3) {$B$};
	  \node[color=red] at (s-2-2) {$X$};
	  \node[color=red] at (s-2-4) {$Y$};
      }
      
      
      \visible<5-6>{
	\path[->,color=red,thick] (s-3-3) edge (s-2-2);
	\path[->,color=red,thick] (s-3-3) edge (s-2-4);
	\node[color=red] at (s-3-3) {$X \times_B Y$};
      }
      
      \visible<7->{
	\node at (s-2-2) {$X$};
	\node at (s-2-4) {$Y$};
	\node[color=gray] at (s-1-3) {$B$};
	\node at (s-3-3) {$X \times_B Y$};
	\path[->,color=gray] (s-2-2) edge (s-1-3);
	\path[->,color=gray] (s-2-4) edge (s-1-3);
	\path[->,thick,thick] (s-2-4) edge (s-1-5);  
	\path[->,thick,thick] (s-2-2) edge (s-1-1);
	\path[->,thick,thick] (s-3-3) edge (s-2-2);
	\path[->,thick,thick] (s-3-3) edge (s-2-4);
      }
   \end{tikzpicture}
   \end{center}
 \end{overlayarea}
 \end{block}
 \visible<8->{
 \begin{center}
  $\rightsquigarrow$ Category of generalized morphisms $\G( \mathbf{A} )$
 \end{center}}
 
\end{frame}

% Pseudo-inverses
\begin{frame}[fragile]
  \frametitle{Pseudo-inverses}
 \begin{block}{Pseudo-inverses}
  \begin{center}
  \begin{tabular}{ccc}
   \begin{tikzpicture}[baseline=(s-2-2)]
      \matrix[matrix of math nodes,column sep={20pt,between origins},row sep={25pt,between origins}] (s)
      {
      A & & &  & B\\
      & \phantom{D} & & \phantom{D} & \\
      & & D  &  & \\
      };
         \visible<1>{
         \path[->, thick] (s-3-3) edge (s-1-1);
         \path[->, thick] (s-3-3) edge (s-1-5);
         }
         \visible<2->{
         \path[->,red, thick] (s-3-3) edge (s-1-1);
         \path[->,blue, thick] (s-3-3) edge (s-1-5);
         }
         \visible<3->{\path[<->,bend left, thick] (s-2-2) edge (s-2-4);}
         
         \path[->,dashed, thick] (s-1-1) edge (s-1-5);
    \end{tikzpicture}
    
    &
    \pause \pause \pause  {$\rightsquigarrow$} \pause
    &
      \begin{tikzpicture}[baseline=(s-2-2)]
      \matrix[matrix of math nodes,column sep={20pt,between origins},row sep={25pt,between origins}] (s)
      {
      B & & &  & A\\
      & \phantom{D} & & \phantom{D} & \\
      & & D  &  & \\
      };
         \path[->,blue, thick] (s-3-3) edge (s-1-1);
         \path[->,red, thick] (s-3-3) edge (s-1-5);
         \path[->,dashed, thick] (s-1-1) edge (s-1-5);
    \end{tikzpicture}
   
   
  \end{tabular}
\end{center}
 \end{block}
\end{frame}

% Honest morphisms
\begin{frame}[fragile]
 \frametitle{Honest morphisms}
 \pause
 \begin{block}{Honest morphisms}
  $\mathbf{A}$ embeds into $\G( \mathbf{A} )$:
  \pause
  \begin{center}
   \begin{tabular}{ccc}
    \begin{tikzpicture}[baseline=(base)]
        \coordinate (r) at (2.5,0);
        \coordinate (d) at (0,-2);
        \node (A) {$A$};
        \node (B) at ($(A) + (r)$) {$B$};
        \node (base) at ($(A) + (0,-0.1)$) {};
        \draw[->,thick,blue] (A) -- (B);
    \end{tikzpicture}
    
    &
    \pause {$\mapsto$} \pause
    &
      \begin{tikzpicture}[baseline=(s-2-2)]
      \matrix[matrix of math nodes,column sep={20pt,between origins},row sep={25pt,between origins}] (s)
      {
      A & & &  & B\\
      & \phantom{A} & & \phantom{A} & \\
      & & A  &  & \\
      };
         \path[->, thick] (s-3-3) edge node[left]{$\id_A$} (s-1-1);
         \path[->,blue, thick] (s-3-3) edge (s-1-5);
         \path[->,dashed, thick] (s-1-1) edge (s-1-5);
    \end{tikzpicture}
  \end{tabular}
  \end{center}
  \pause
  Generalized morphisms with such a representation are called \textbf{honest}.
 \end{block}
\end{frame}

% 
\begin{frame}[fragile]
 \frametitle{Honest morphisms}
 \begin{itemize}
  \item[Q:] When does $A \stackrel{\alpha}{\longleftarrow} D \stackrel{\beta}{\longrightarrow} B$ define an honest morphism?
 \end{itemize}
 \pause
 \begin{block}{Domain}
  $\domain(A \stackrel{\alpha}{\longleftarrow} D \stackrel{\beta}{\longrightarrow} B) := \im( \alpha ) \visible<5->{{\color{red}\ = A }}$
 \end{block}
 \pause
 \begin{block}{Defect}
  $\defect(A \stackrel{\alpha}{\longleftarrow} D \stackrel{\beta}{\longrightarrow} B) := \beta( \kernel( \alpha ) ) \visible<7->{{\color{red}\ = 0 }}$
 \end{block}
 \pause
 \begin{itemize}
  \item[A:] When it has a full domain \pause\pause and $0$ defect.
 \end{itemize}

\end{frame}

% Computing representatives
\begin{frame}[fragile]
 \frametitle{Computing representatives}
 \pause
 Given an honest generalized morphism in $\G( \mathbf{A} )$, compute the corresponding morphism in $\mathbf{A}$.
 \pause
 \begin{block}{}
  \begin{center}
   \begin{tikzpicture}[baseline=(s-2-2)]
      \matrix[matrix of nodes,column sep={20pt,between origins},row sep={20pt,between origins}] (s)
      {
        & & |[name=P]|\text{\visible<4-5>{{\only<5>{\color{gray}}$A\sqcup_D B$}}} & &\\
        & &   & &\\
      |[name=A]|$A$ & & &  & |[name=B]|$B$\\
      &  & &  & \\
      & & |[name=D]| \text{\only<1-4>{{\only<4>{\color{gray}}$D$}} \only<5-7>{$A \times_{A\sqcup_D B} B$}}  &  & \phantom{$A \times_{A\sqcup_D B} B$} \\
      };
         \visible<1-3,5-7>{
         \path[->, thick] (D) edge (B);
         }
         \visible<1-3,5>{
         \path[->, thick] (D) edge (A);
         }
         \visible<4>{
         \path[->, thick,gray] (D) edge (B);
         }
         \visible<4>{
         \path[->, thick,gray] (D) edge (A);
         }
         
         \visible<6>{
         \path[->,thick] (D) edge node[above, inner sep = 0em,rotate=-45]{$\sim$} (A);
         }
         \visible<7>{
         \path[->,thick] (A) edge node[above, inner sep = 0em,rotate=-45]{$\sim$} (D);
         }
         \visible<8>{
         \path[->,thick] (A) edge (B);
         }
         
         \visible<4>{
         \path[->,thick] (A) edge (P);
         \path[->,thick] (B) edge (P);
         }
         \visible<5>{
         \path[->,thick,gray] (A) edge (P);
         \path[->,thick,gray] (B) edge (P);
         }
         
         
    \end{tikzpicture}
  \end{center}
 \end{block}
\end{frame}

% Computing representatives
\begin{frame}[fragile]
 \frametitle{Computing representatives}
 Given an honest generalized morphism in $\G( \mathbf{A} )$, compute the corresponding morphism in $\mathbf{A}$.
 \begin{block}{}
  \begin{center}
   \begin{tikzpicture}[baseline=(s-2-2)]
      \matrix[matrix of nodes,column sep={20pt,between origins},row sep={20pt,between origins}] (s)
      {
        & & |[name=P]|\text{\visible<2,3>{{\only<3>{\color{gray}}$\Q$}}} & & \phantom{$A\sqcup_D B$}\\
        & &   & &\\
      |[name=A]|$\Q$ & & &  & |[name=B]|$\Q$\\
      &  & &  & \\
      & & |[name=D]| \text{\only<1,2>{{\only<2>{\color{gray}}$\Q^2$}} \only<3-5>{$\Q$}}  &  & \phantom{$A \times_{A\sqcup_D B} B$} \\
      };
         \visible<1>{
         \path[->, thick] (D) edge (B);
         \path[->, thick] (D) edge (A);
         }
         \visible<2>{
         \path[->, thick,gray] (D) edge (B);
         \path[->, thick,gray] (D) edge (A);
         }
         \visible<3-5>{
         \path[->, thick] (D) edge node[right, inner sep = 0.5em] {$2$} (B);
         }
         \visible<3>{
         \path[->, thick] (D) edge node[left, inner sep = 0.5em]{$1$} (A);
         }
         \visible<4>{
         \path[->,thick] (D) edge node[left, inner sep = 0.5em]{$1$} node[above, inner sep = 0em,rotate=-45]{$\sim$} (A);
         }
         \visible<5>{
         \path[->,thick] (A) edge node[left, inner sep = 0.5em]{$1$} node[above, inner sep = 0em,rotate=-45]{$\sim$} (D);
         }
         \visible<6>{
         \path[->,thick] (A) edge node[above]{$2$} (B);
         }
         
         \visible<2>{
         \path[->,thick] (A) edge node[left, inner sep = 0.5em] {$2$} (P);
         \path[->,thick] (B) edge node[right, inner sep = 0.5em] {$1$}(P);
         }
         \visible<3>{
         \path[->,thick,gray] (A) edge node[left, inner sep = 0.5em,gray] {$2$} (P);
         \path[->,thick,gray] (B) edge node[right, inner sep = 0.5em,gray] {$1$}(P);
         }
         
         \visible<1>{
         \node[above,font=\tiny] at ($(A) + (0,-1.2)$) {$\left(\begin{array}{cc} 1 \\ -1 \end{array} \right)$};
         \node[above,font=\tiny] at ($(B) + (0,-1.2)$) {$\left(\begin{array}{cc} 2 \\ -2 \end{array} \right)$};
         }
         \visible<2>{
         \node[above,font=\tiny,gray] at ($(A) + (0,-1.2)$) {$\left(\begin{array}{cc} 1 \\ -1 \end{array} \right)$};
         \node[above,font=\tiny,gray] at ($(B) + (0,-1.2)$) {$\left(\begin{array}{cc} 2 \\ -2 \end{array} \right)$};
         }
         
         
    \end{tikzpicture}
  \end{center}
 \end{block}
\end{frame}

% Constructive diagram chases
\begin{frame}[fragile]
 \frametitle{Constructive diagram chases}
 \pause
 \begin{block}{Strategy for constructive diagram chases}
  \pause
  \begin{enumerate}
   \item Compute in $\G( \mathbf{A} )$ using pseudo-inverses and compositions.
   \pause
   \item Compute the honest representative of the resulting generalized morphism.
  \end{enumerate}
 \end{block}
\end{frame}

\begin{frame}[fragile]
 \frametitle{Example: functoriality of homology}
 Let $\left( P_\bullet, \partial \right)$ be a complex in an abelian category $\mathcal{A}$. \pause Then we can compute the generalized
 embedding of the $i$-th homology. \pause
 \begin{block}{}
 \begin{center}
   \begin{tikzpicture}
      \matrix (s) [matrix of math nodes,column sep=20pt,row sep=20pt,nodes in empty cells]
      {  P_{i+1} & & P_i & & P_{i-1} \\
         & \phantom{\Image \left( \partial_{i+1} \right)} & & \phantom{\Kernel \left( \partial_i \right)} & \phantom{\CH_i \left( P_\bullet \right)} \\  };
      \path[->,thick] (s-1-1) edge node[above]{$\partial_{i+1}$} (s-1-3);
      \path[->,thick] (s-1-3) edge node[above]{$\partial_{i}$} (s-1-5);
      
      \visible<4->{
        \node at (s-2-2) {$\Image \left( \partial_{i+1} \right)$};
        \path[right hook->,thick] (s-2-2) edge (s-1-3);
      }
      
      \visible<5->{
        \node at (s-2-4) {$\Kernel \left( \partial_i \right)$};
      }
      
      \visible<5-7>{
        \path[right hook->,thick] (s-2-4) edge (s-1-3);
      }
      
      \visible<6->{
        \path[right hook->,thick] (s-2-2) edge (s-2-4);
      }
      
      \visible<7->{
        \node at (s-2-5) {$\CH_i \left( P_\bullet \right)$};
      }
      
      \visible<7>{
        \path[->>,thick] (s-2-4) edge (s-2-5);
      }
      
      \visible<8->{
        \path[->>,color=red,thick] (s-2-4) edge (s-2-5);
        \path[right hook->,color=red,thick] (s-2-4) edge (s-1-3);
        \path[left hook->,dashed,thick] (s-2-5) edge (s-1-3);
      }
      
   \end{tikzpicture}
  \end{center}
 \end{block}
\end{frame}

\begin{frame}[fragile]
 \frametitle{Example: functoriality of homology}
 
 \begin{theorem}
  Let $\mathcal{A}$ be an abelian category and $\varepsilon: P_\bullet \rightarrow Q_\bullet$ a chain morphism. \pause
  Then the morphism $\CH_i \left( P_\bullet \right) \rightarrow \CH_i \left( Q_\bullet \right)$ can be computed using generalized morphisms: \pause
  \begin{center}
   \begin{tikzpicture}
      \matrix (s) [matrix of math nodes,column sep=20pt,row sep=20pt,nodes in empty cells]
      {  \phantom{\CH_i \left( P_\bullet \right)} & P_i & Q_i & \phantom{ \CH_i \left( Q_\bullet \right) } \\ };
      
      \path[->,thick] (s-1-2) edge node[above] {$\varepsilon_i$} (s-1-3);
      
      \visible<4->{
        \node at (s-1-1) {$\CH_i \left( P_\bullet \right)$};
        \path[->,dashed,thick] (s-1-1) edge (s-1-2);
      }
      
      \visible<5->{
        \node at (s-1-4) {$\CH_i \left( Q_\bullet \right)$};
      }
      \visible<5>{
        \path[->,dashed,thick] (s-1-4) edge (s-1-3);
      }
      
      \visible<6->{
        \path[->,dashed,color=red,thick] (s-1-3) edge (s-1-4);
      }
      \visible<7->{
        \path[->, bend right,thick] (s-1-1) edge (s-1-4);
      }
   \end{tikzpicture}
  \end{center}
 \end{theorem}
\end{frame}


\section{Applications of generalized morphisms}


\subsection{An algorithm for spectral sequences}


\begin{frame}[fragile]
  \frametitle{Spectral sequences via generalized morphisms}
  \begin{center}
    \begin{tabular}{c}
      \vspace{-1em}
      \only<1>{Given: an excerpt of a filtered chain complex.}
      \only<2>{We pass to its graded parts.}
      \only<3-6>{We can compute the differentials via generalized morphisms.}
      \only<7>{This is a generalized \textbf{subquotient embedding}.}
      \only<11>{This is a generalized \textbf{subquotient projection}.}
      \only<12-14>{We can compose the arrows.}
      \only<15-16>{This formula still makes sense if we map $1$ step down.}
      \only<17->{One more step \dots}
      \\
      \hspace{30em}
      \\[-0.1em]
      \hline   
      \begin{tikzpicture}[every node/.style={scale=0.8},baseline=(S1)]
            \coordinate (r) at (1.75,0);
            \coordinate (d) at (0,-1.1);
            \node (A1) {};
            \node (A2) at ($(A1) + (r)$) {$\vdots$};
            \node (A3) at ($(A2) + (r)$) {$\vdots$};
            \node (A4) at ($(A3) + (r)$) {$\vdots$};
            \node (A5) at ($(A4) + (r)$) {};
            
            \node (B1) at ($(A1) + 0.9*(d)$) {$\cdots$};
            
            \node (B2) at ($(B1) + (r)$) {\only<1>{$A_{i+1}$}\only<2->{$\frac{A_{i+1}}{A_{i}}$}};

            \node (B3) at ($(B2) + (r)$) {\only<1>{$B_{i+1}$}\only<2->{$\frac{B_{i+1}}{B_{i}}$}};
            \node (B4) at ($(B3) + (r)$) {\only<1>{$C_{i+1}$}\only<2->{$\frac{C_{i+1}}{C_{i}}$}};
            \node (B5) at ($(B4) + (r)$) {$\cdots$};
            
            \node (C1) at ($(B1) + (d)$) {$\cdots$};
            \node (C2) at ($(C1) + (r)$) {\only<1>{$A_{i}$}\only<2->{$\frac{A_{i}}{A_{i-1}}$}};
            \node (C3) at ($(C2) + (r)$) {\only<1>{$B_{i}$}\only<2->{$\frac{B_{i}}{B_{i-1}}$}};
            \node (C4) at ($(C3) + (r)$) {\only<1>{$C_{i}$}\only<2->{$\frac{C_{i}}{C_{i-1}}$}};
            \node (C5) at ($(C4) + (r)$) {$\cdots$};
            
            \node (D1) at ($(C1) + (d)$) {$\cdots$};
            \node (D2) at ($(D1) + (r)$) {\only<1>{$A_{i-1}$}\only<2->{$\frac{A_{i-1}}{A_{i-2}}$}};
            \node (D3) at ($(D2) + (r)$) {\only<1>{$B_{i-1}$}\only<2->{$\frac{B_{i-1}}{B_{i-2}}$}};
            \node (D4) at ($(D3) + (r)$) {\only<1>{$C_{i-1}$}\only<2->{$\frac{C_{i-1}}{C_{i-2}}$}};
            \node (D5) at ($(D4) + (r)$) {$\cdots$};
            
            \node (E1) at ($(D1) + 0.9*(d)$) {};
            \node (E2) at ($(E1) + (r)$) {$\vdots$};
            \node (E3) at ($(E2) + (r)$) {$\vdots$};
            \node (E4) at ($(E3) + (r)$) {$\vdots$};
            \node (E5) at ($(E4) + (r)$) {};
            
            \node (S1) at ($(A1) - 0.5*(d)$) {$\cdots$};
            \node (S2) at ($(A2) - 0.5*(d)$) {$A$};
            \node (S3) at ($(A3) - 0.5*(d)$) {$B$};
            \node (S4) at ($(A4) - 0.5*(d)$) {$C$};
            \node (S5) at ($(A5) - 0.5*(d)$) {$\cdots$};
            
            \only<1-15>{
                \draw[->,thick] (B1) -- (B2);
                \draw[->,thick] (B2) --node[above]{\only<2-15>{$\overline{\partial}$}} (B3);
                \draw[->,thick] (B3) --node[above]{\only<2-15>{$\overline{\partial}$}} (B4);
                \draw[->,thick] (B4) -- (B5);
                
                \draw[->,thick] (C1) -- (C2);
                \draw[->,thick] (C2) -- (C3);
                \draw[->,thick] (C3) -- (C4);
                \draw[->,thick] (C4) -- (C5);
                
                \draw[->,thick] (D1) -- (D2);
                \draw[->,thick] (D2) -- (D3);
                \draw[->,thick] (D3) -- (D4);
                \draw[->,thick] (D4) -- (D5);
            }
            
            \draw[->,thick] (S1) -- (S2);
                \draw[->,thick] (S2) --node[above]{$\partial$} (S3);
                \draw[->,thick] (S3) --node[above]{$\partial$} (S4);
                \draw[->,thick] (S4) -- (S5);
            
            \only<16-17>{
                \draw[->,thick,dashed] (B1) -- (C2);
                \draw[->,thick,dashed] (B2) --node[above]{$\overline{\partial}^1$} (C3);
                \draw[->,thick,dashed] (B3) -- (C4);
                \draw[->,thick,dashed] (B4) -- (C5);
                
                \draw[->,thick,dashed] (C1) -- (D2);
                \draw[->,thick,dashed] (C2) -- (D3);
                \draw[->,thick,dashed] (C3) --node[above]{$\overline{\partial}^1$} (D4);
                \draw[->,thick,dashed] (C4) -- (D5);
            }
            
            \only<18->{
                \draw[->,thick,dashed] (B1) -- (D2);
                \draw[->,thick,dashed] (B2) --node[above,xshift=0.3em]{$\overline{\partial}^2$} (D3);
                \draw[->,thick,dashed] (B3) -- (D4);
                \draw[->,thick,dashed] (B4) -- (D5);
            }
            
            
            \only<1>{
              \draw[right hook->,thick] (B2) -- (A2);
              \draw[right hook->,thick] (B3) -- (A3);
              \draw[right hook->,thick] (B4) -- (A4);
              
              \draw[right hook->,thick] (C2) -- (B2);
              \draw[right hook->,thick] (C3) -- (B3);
              \draw[right hook->,thick] (C4) -- (B4);
              
              \draw[right hook->,thick] (D2) -- (C2);
              \draw[right hook->,thick] (D3) -- (C3);
              \draw[right hook->,thick] (D4) -- (C4);
              
              \draw[right hook->,thick] (E2) -- (D2);
              \draw[right hook->,thick] (E3) -- (D3);
              \draw[right hook->,thick] (E4) -- (D4);
            }
        \end{tikzpicture}
      
      

      \end{tabular}
    
    \end{center}
    \vspace{-1em}
    \visible<3->{
     \begin{block}{}
     
      \begin{center} 
      
        \begin{tikzpicture}[every node/.style={scale=0.8}]
            \coordinate (r) at (1.75,0);
            \coordinate (d) at (0,-1.1);
            \visible<4->{
                \node (A1) {$\frac{A_{i+1}}{A_{i}}$};
            }
            
            \visible<5->{
                \node (A2) at ($(A1) + (r)$) {$A_{i+1}$};
            }
            \visible<6->{
                \node (A3) at ($(A2) + (r)$) {$A$};
            }
            \visible<8->{
                \node (A4) at ($(A3) + (r)$) {$B$};
            }
            
            \visible<9->{
                \node (A5) at ($(A4) + (r)$) {\only<4-15>{$B_{i+1}$}\only<16-17>{$B_{i}$}\only<18->{$B_{i-1}$}};
            }
            
            \visible<4->{
                \node (A6) at ($(A5) + (r)$) {\only<4-15>{$\frac{B_{i+1}}{B_{i}}$}\only<16-17>{$\frac{B_{i}}{B_{i-1}}$}\only<18->{$\frac{B_{i-1}}{B_{i-2}}$}};
            }
            
            \visible<3->{
                \node (S) at ($(A1) - 0.4*(r)$) {\only<1-15>{$\overline{\partial}^{\phantom{1}}:$}\only<16-17>{$\overline{\partial}^1:$}\only<18->{$\overline{\partial}^2:$}  };
            }
            
            \visible<5->{
                \draw[->>,thick] (A2) -- (A1);
            }
            
            \visible<6->{
                \draw[right hook->,thick] (A2) -- (A3);
            }
            
            \visible<8->{
                {
                \only<13->{\color{red}}
                \draw[->,thick] (A3) --node[above]{$\partial$} (A4);
                }
            }
            
            \visible<9->{
                \draw[left hook->,thick] (A5) -- (A4);
            }
            
            \visible<10->{
                \draw[->>,thick] (A5) -- (A6);
            }
            \visible<7->{
                {
                \only<12->{\color{red}}
                \draw[bend left,->,thick,out=25,in=155, dashed] (A1) to (A3);
                }
            }
            \visible<11->{
                {
                \only<14->{\color{red}}
                \draw[bend left,->,thick,out=25,in=155, dashed] (A4) to (A6);
                }
            }
        \end{tikzpicture}
      
      \end{center}
    
     \end{block}
    }

\end{frame}


\begin{frame}[fragile]
 \frametitle{Spectral sequences via generalized morphisms}
 
 For all $r \geq 0$, we get so-called generalized chain complexes.
 
 \begin{block}{}
    \begin{overlayarea}{\linewidth}{8\baselineskip}
    \begin{center}
        
        \begin{tikzpicture}[every node/.style={scale=0.8}]
                \coordinate (r) at (3,0);
                \coordinate (d) at (0,-1.5);
                \node (A1) {$\dots$};
                \node (A2) at ($(A1) + 0.6*(r)$) {\only<1-5>{$\frac{A_{i+1}}{A_{i}}$}\only<6->{$\frac{F_{i+1}C_{j+1}}{F_{i}C_{j+1}}$} };
                \node (A3) at ($(A2) + (r)$) {\only<1-5>{$\frac{B_{i+1-r}}{B_{i-r}}$}\only<6->{$\frac{F_{i+1-r}C_{j}}{F_{i-r}C_{j}}$}};
                \node (A4) at ($(A3) + (r)$) {\only<1-5>{$\frac{C_{i+1-2r}}{C_{i-2r}}$}\only<6->{$\frac{F_{i+1-2r}C_{j-1}}{F_{i-2r}C_{j-1}}$}};
                \node (A5) at ($(A4) + 0.6*(r)$) {$\dots$};
                
                \draw[->,thick,dashed] (A1) -- (A2);
                \draw[->,thick,dashed] (A2) --node[above]{\only<1-5>{$\overline{\partial^r_A}$}} (A3);
                \draw[->,thick,dashed] (A3) --node[above]{\only<1-5>{$\overline{\partial^r_B}$}} (A4);
                \draw[->,thick,dashed] (A4) -- (A5);
                
                \visible<6->{
                \node (C1) at ($(A1) - (d)$) {$\dots$};
                \node (C2) at ($(C1) + 0.6*(r)$) {$C_{j+1}$};
                \node (C3) at ($(C2) + (r)$) {$C_{j}$};
                \node (C4) at ($(C3) + (r)$) {$C_{j-1}$};
                \node (C5) at ($(C4) + 0.6*(r)$) {$\dots$};
                \draw[->,thick](C1) -- (C2);
                \draw[->,thick](C2) -- (C3);
                \draw[->,thick](C3) -- (C4);
                \draw[->,thick](C4) -- (C5);
                \draw[->,thick,dashed](A2) -- (C2);
                \draw[->,thick,dashed](A3) -- (C3);
                \draw[->,thick,dashed](A4) -- (C4);
                }
                
                \visible<2->{
                  {
                      \only<5->{\color{red}}
                      \node (B3) at ($(A3) + (d)$) {\only<2-5>{$\frac{\domain( \overline{\partial^r_B} ) }{\defect( \overline{\partial^r_A} )}$}\only<6->{$E^r_{i+1-r,j-i+(r-1)}$}};
                  }
                  \draw[->,thick,dashed] (B3) -- (A3);
                }
                
                \visible<3->{
                    
                    {
                        \only<5->{\color{red}}
                        \node (B1) at ($(A1) + (d)$) {$\dots$};
                        \node (B2) at ($(A2) + (d)$) {\only<3-5>{$\frac{\domain}{\defect}$}\only<6->{$E^r_{i+1,j-i}$}};
                        
                        \node (B4) at ($(A4) + (d)$) {\only<3-5>{$\frac{\domain}{\defect}$}\only<6->{$E^r_{i+1-2r,j-i+2(r-1)}$}};
                        \node (B5) at ($(A5) + (d)$) {$\dots$};
                    }
                    
                    \draw[->,thick,dashed] (B2) -- (A2); 
                    \draw[->,thick,dashed] (B4) -- (A4);
                }
                
                \visible<4->{
                  {
                    \only<5->{\color{red}}
                    \draw[->,thick] (B1) -- (B2);
                    \draw[->,thick] (B2) -- (B3);
                    \draw[->,thick] (B3) -- (B4);
                    \draw[->,thick] (B4) -- (B5);
                  }
                }
        \end{tikzpicture}
        
    \end{center}
    \end{overlayarea}
 \end{block}
 \visible<5->{
    \begin{block}{}
        \begin{itemize}
         \item These are the chain complexes on the $r$-th page of the associated \textbf{spectral sequence}.
         \visible<7->{
         \item We just computed them \textbf{without a recursive strategy}.
         }
        \end{itemize}

    \end{block}

 }
\end{frame}


\subsection{The purity filtration}



\begin{frame}[fragile]
\frametitle{Spectral sequences}

\begin{block}{Convergence}
Let $C_\bullet := 0= F_{-n-1}C_\bullet \leq F_{-n} C_\bullet \leq \cdots \leq F_0 C_\bullet$ be a finitely filtered complex \pause
and for $p,q \in \Z$, $r \geq 0$ let $E_{pq}^r$ be the \textbf{computed} objects \pause with their generalized embeddings
   \begin{center}
   \begin{tikzpicture}
      \matrix (s) [matrix of math nodes,column sep=50pt,row sep=20pt]
      {  E^r_{pq} & C_{p+q}. \\ };
      
      \path[right hook->, dashed,thick] (s-1-1) edge (s-1-2);
   \end{tikzpicture}
  \end{center} \pause
 Then for all $p,q$ and all $k \geq n+1$ we have \pause
 \[
  E^k_{pq} \cong E^{n+1}_{pq} \visible<6->{ =:E^{\infty}_{pq}.}
 \]
\end{block}
\end{frame}


\begin{frame}[fragile]
 \frametitle{Filtration morphisms}
 %für filtrierte complexe
 \begin{block}{}
  Using the above generalized embedding and the generalized projection to the homology, \pause we get a generalized morphism \pause
  \begin{center}
      \begin{tikzpicture}
      \matrix (s) [matrix of math nodes,column sep=35pt,row sep=30pt]
      {  E_{p,q}^\infty & \phantom{C_{p+q}^a} & \phantom{\CH_{p+q} \left( C_\bullet \right) } \\
        & \phantom{X} & \\};
      \node at (s-1-3) {$\CH_{p+q} \left( C_\bullet \right)$};
      \visible<3-6>{
	\path[->,dashed, bend right,thick] (s-1-1) edge (s-1-3);
      }
      \visible<4-6>{
	\path[right hook->, dashed,thick] (s-1-1) edge (s-1-2);
	\node at (s-1-2) {$C_{p+q}$};
      }
      
      \visible<5-6>{
	\path[->>,dashed,thick] (s-1-2) edge (s-1-3);
      }
      \visible<7->{
        \node at (s-2-2) {$X$};
        \path[->,thick] (s-2-2) edge node[left,inner sep = 1em] {$\alpha$} (s-1-1);
        \path[->,thick] (s-2-2)  edge node[right,inner sep = 1em] {$\beta$} (s-1-3);
     }
   \end{tikzpicture}
  \end{center}
  \pause \pause \pause \pause \pause
  This induces a filtration on $H := \CH_{p+q} \left( C_\bullet \right)$ with \pause
  \begin{align*}
   F_{p} H /F_{p-1} H  & \cong E_{p,q}^\infty \\
   F_{p} H & \cong \Image \left( \beta \right).
  \end{align*}
 \end{block}
\end{frame}



\begin{frame}
 \frametitle{The bidualizing spectral sequence}
 \begin{theorem}
  Let $M$ be a finitely presented module over a computable ring $S$ of finite projective dimension. \pause
  Then one can \textbf{compute} a filtered complex $C_\bullet$ with the following properties: \pause
  \begin{enumerate}
   \item $C_\bullet$ is exact everywhere except at $0$. \pause
   \item $\CH_0 \left( C_\bullet \right)$ is constructively isomorphic to $M$. \pause
   \item The induced spectral sequence of $C_\bullet$ is the bidualizing spectral sequence, \pause i.e., we have
    \[
  E^2_{pq} = \mathrm{Ext}^{-p}( \mathrm{Ext}^q( M, S ), S ) )  \Longrightarrow
  M \quad \mbox{ for } p+q = 0,
  \] \pause
  which yields the purity filtration of $M$, \pause i.e., a finite filtration where
all graded parts $F_{-i}M / F_{-(i+1)}M$ are pure of codimension $i$.
  \end{enumerate}
 \end{theorem}
\end{frame}

\begin{frame}[fragile]
 \frametitle{Filtration morphisms}
 %für filtrierte complexe
 \begin{block}{}
  Using all of the above, we now get the generalized morphism \pause
  \begin{center}
      \begin{tikzpicture}
      \matrix (s) [matrix of math nodes,column sep=20pt,row sep=30pt]
      {  E_{-p,p}^\infty & \phantom{C_{0}^a} & & \phantom{\CH_0 \left( C_\bullet \right) } & \phantom{M_0^a} \\
        & & \phantom{X} & & \\};
       \node at (s-1-5) {$M$};
      \visible<2-5>{
	\path[->,dashed, bend right,thick] (s-1-1) edge (s-1-5);
      }
      \visible<3-5>{
	\path[right hook->, dashed,thick] (s-1-1) edge (s-1-2);
	\node at (s-1-2) {$C_0$};
      }
      
      \visible<4-5>{
	\path[->>,dashed,thick] (s-1-2) edge (s-1-4);
	\node at (s-1-4) {$\CH_0 \left( C_\bullet \right)$};
      }
      
      \visible<5>{
	\path[->,thick] (s-1-4) edge node[above,inner sep = -0.1em] {$\sim$} (s-1-5);
      }
      \visible<6->{
        \node at (s-2-3) {$X$};
        \path[->,thick] (s-2-3) edge node[left,inner sep = 1em] {$\alpha$} (s-1-1);
        \path[->,thick] (s-2-3)  edge node[right,inner sep = 1em] {$\beta$} (s-1-5);
     }
   \end{tikzpicture}
  \end{center}
  \pause \pause \pause \pause \pause
  For the purity filtration of $M$, we have \pause
  \begin{align*}
   F_{-p} M/F_{-p-1}M  & \cong E_{-p,p}^\infty \\
   F_{-p}M & \cong \Image \left( \beta \right).
  \end{align*}
 \end{block}
\end{frame}


\begin{frame}
 \frametitle{Presentations from filtrations}
 Let $F_{-n} M \leq F_{-n+1} M \leq \dots \leq F_0 M := M$ be a finitely presented filtered module. \pause
 \begin{block}{}
  If $\mathtt{M}_i$ is a presentation matrix for $F_i M / F_{i-1}M$, \pause then $M$ can be presented by
  an upper block triangular matrix
  \[
   \left( \begin{array}{ccccc}
           \mathtt{M}_0 & * & \dots & \dots & * \\
           & \mathtt{M}_{-1} & * & \dots & * \\
           & & \ddots & \ddots & \vdots \\
           & & & \mathtt{M}_{-n+1} & * \\
           & & & & \mathtt{M}_{-n} \\
          \end{array} \right).
  \]
 \end{block}

\end{frame}

\begin{frame}[fragile]
 \frametitle{Example: filtered presentation}
 Consider the module with relations
\begin{small}
\[
\left(
  \begin{array}{cccccc}
0&0&                 0&    0&       xz&-z^2 \\   
0&    0&                 0&    0&       xy&-yz\\
0&    -x^2z+xyz+xz^2&y^2z&-xz+yz&x-y&0\\   
0&0&                 0&    0&       x^2&-xz\\
-xy&    -x^3+x^2y+x^2z&  xy^2&-x^2+xy&0&  x-y\\
 z& 0&                 0&    0&       0&  0\\ 
  \end{array}
\right)
\]
\end{small}
\pause Computing the purity filtration by using the bidualizing spectral sequence yields
\begin{small}
\[
 \left(
 \begin{array}{ccccc|c|c}
x& -z& 0&    0&     0&       0&  1\\ 
-y&z&  y^2z&-yz^2&-xz+yz&0&  -1\\
0& x-y&xy^2&-xyz&-x^2+xy&xy&0\\
\hline
0& 0&  0&    0&     0&       z&  0\\ 
\hline
0& 0&  0&    0&     0&       0&  z\\ 
0& 0&  0&    0&     0&       0&  y\\ 
0& 0&  0&    0&     0&       0&  x \\
\end{array} \right)
\]
\end{small}
\end{frame}
